\section{Formulation 18 (CAPTURE POINT RELAXATION)}\label{sec.form05}

%%%%%%%%%%%%%%%%%%%%%%%%%%%%%%%%%%%%%%%%%%%%%%%%%%%%%%%%%%%%%%%%%%%%%%%%%%%%%%%%%%%%%%%%%%%%%%%%%%%%%%

\subsection{Description}
\begin{itemize}
    \item Formulation 5 with relaxed terminal constraints on capture point.
    \item Two QP problems with different objective functions are solved sequentially.
\end{itemize}


%%%%%%%%%%%%%%%%%%%%%%%%%%%%%%%%%%%%%%%%%%%%%%%%%%%%%%%%%%%%%%%%%%%%%%%%%%%%%%%%%%%%%%%%%%%%%%%%%%%%%%

\subsection{First QP}

Decision variables:
\begin{equation*}
    \V{X} = 
    \begin{bmatrix}
        \cJerk \\
        \FD \\
        \V{w} \\
    \end{bmatrix}
\end{equation*}

Objective function:
\begin{equation*}
\begin{split}
    \minimize{\V{X}}    & \norm{\V{w}}^2
\end{split}
\end{equation*}

Inequality constraints are the same as in Formulation 5.

The terminal equality constraints for the capture point are defined as:
\begin{equation*}
    \zmp_N = \cp - \V{w}
\end{equation*}

\begin{align*}
    \begin{bmatrix}
        \frac{1}{\omega}\M{D}_{cpt} \M{U}_e     &   \M{0}   &   -\M{I}\\
    \end{bmatrix}
    \V{X}
    = -\frac{1}{\omega}\M{D}_{cpt} \M{S}_e \cstate_0
\end{align*}


%%%%%%%%%%%%%%%%%%%%%%%%%%%%%%%%%%%%%%%%%%%%%%%%%%%%%%%%%%%%%%%%%%%%%%%%%%%%%%%%%%%%%%%%%%%%%%%%%%%%%%

\subsection{Second QP}
The decision variables, objective function, and inequality constraints are
the same as in the Formulation 5.

The terminal equality constraints are
\begin{equation*}
    \zmp_N = \cp - \V{w}^{*},
\end{equation*}
where $\V{w}^{*}$ is taken from the solution of the first QP.

Consequently
\begin{align*}
    \begin{bmatrix}
        \frac{1}{\omega}\M{D}_{cpt} \M{U}_e     &   \M{0}   &   -\M{I}\V{w}^{*}\\
    \end{bmatrix}
    \V{X}
    = -\frac{1}{\omega}\M{D}_{cpt} \M{S}_e \cstate_0
\end{align*}
